\subsection{1.}

  \begin{itemize}
    \item a.
        os valores de $I_{DQ}$ e $V_{DQ}$ é o valor da curva na interseção
        com a reta de carga.
        A reta é dada por :

        $$
          I_D = \frac{E}{R} = 0.02 A

        $$
        \begin{itemize}
        \item Em que E é a tensão fornecida pela fonte.
        \item E R a resistencia do  circuito
        \end{itemize}

        $$
        V_D = \left. E \right|_{I_D = 0} = 10 V
        $$
        Da analise gráfica percebemos que o valor de $I_{DQ}$ = 18.5 mA portanto
        podemos presumir que $V_{DQ}$ = 0.78 V



        V_{DQ} =
    \item b.
        $$
        V_R = E - V_D = 10 - 0.78 = 9.22 V
        $$
    \item c.
        A resistencia CC é dada por:
          $$
          V_{DQ} = R_{DQ}I_{DQ} => R_{DQ} = \frac{V_{DQ}}{I_{DQ}} = 42.16 \Omega
          $$
